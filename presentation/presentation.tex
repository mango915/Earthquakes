% chktex-file 46
%!TeX spellcheck = en-US,it-IT
\documentclass{beamer}
\usepackage[english]{babel}
\usepackage{filecontents}
\usepackage{eso-pic,graphicx}
\usepackage[top=2cm, bottom=2cm, outer=0cm, inner=0cm]{geometry}
\usepackage[absolute,overlay]{textpos}
\begin{filecontents}{\jobname.bib}
 @ARTICLE{paper,
 author = {{Baldi}, P. and {Sadowski}, P. and {Whiteson}, D.},
 title = "{Searching for exotic particles in high-energy physics with deep learning}",
 journal = {Nature Communications},
 year = 2014
 }
\end{filecontents}

\usepackage[style=authortitle,backend=biber]{biblatex}
\addbibresource{\jobname.bib}
\usetheme{Padova}

\title{Eathquakes}
%\subtitle{Curabitur sit amet mi magna}
\author{Nicola Dainese, Stefano Mancone, Francesco Vidaich}
\date{03 Marzo 2019}
\setbeamertemplate{caption}{\raggedright\insertcaption\par}
\begin{document}

\maketitle

\section{Data exploration}
\begin{frame}{Data exploration}
 \begin{figure}
  \centering
  \includegraphics[width=.49\textwidth]{figures/IDvt.png}
  \includegraphics[width=.49\textwidth]{figures/3Dxyz.png}\\
 \end{figure}
 \AddToShipoutPictureBG{\includegraphics[width=\paperwidth,height=0.08\paperheight]{bgWave.pdf}}
\end{frame}

\begin{frame}{Magnitude distribution}
 \begin{figure}
  \centering
  \includegraphics[width=.7\textwidth]{figures/magn_d.png}\\  \end{figure}
 \begin{itemize}
  \item $f(m) = N_0 e ^ {-am}$
        \begin{itemize}
         \item{$N_0 = 1.1e7\pm$}
         \item{$a = 2.23\pm$}
        \end{itemize}
  \item $\sigma_{bin} \approx 1/ \sqrt{N}$
 \end{itemize}
\end{frame}

\begin{frame}{PCA analysis}
 \begin{textblock*}{6.5cm}(0.5cm,1.5cm) % {block width} (coords)
  \includegraphics[width=1\textwidth]{figures/PCA.png}\\
 \end{textblock*}
 \begin{textblock*}{5cm}(1.5cm,5cm) % {block width} (coords)
  \includegraphics[width=1\textwidth]{figures/3Dplane.png}\\
 \end{textblock*}
 \begin{textblock*}{5cm}(7.5cm,5.3cm) % {block width} (coords)
  \includegraphics[width=1\textwidth]{figures/heat_d.png}\\
 \end{textblock*}
\end{frame}













\section{Data exploration}
\begin{frame}{Data exploration}
 \begin{itemize}
  \item address a signal background discrimination problem with the deep
        neural networks machine learning model.
  \item use of high-performance computational techniques:
        \begin{itemize}
         \item cluster computing
         \item hpc
        \end{itemize}
  \item inspired by \cite{paper} work, 2014.
 \end{itemize}
 \AddToShipoutPictureBG{\includegraphics[width=\paperwidth,height=0.08\paperheight]{bgWave.pdf}}
\end{frame}
\begin{frame}{The Higgs dataset}
 \begin{textblock*}{8cm}(2cm,2cm) % {block width} (coords)
  signal event
 \end{textblock*}
 \begin{textblock*}{8cm}(7.5cm,2cm) % {block width} (coords)
  background event
 \end{textblock*}
 \begin{figure}[htpb]
  \centering
  \includegraphics[width=1\textwidth]{figures/decays.png}
  \caption{the two $gg \rightarrow W^\mp W^\pm b \bar{b}$ processes}
  \label{graphs}
 \end{figure}
 
 \begin{itemize}
  \item semi-leptonic decay mode: $W_1 \rightarrow l\nu$, $W_2 \rightarrow jj $
  \item final products: $l\nu bjjb$
  \item $21$ low-level features, $7$ high-level features
 \end{itemize}
 \AddToShipoutPictureBG{\includegraphics[width=\paperwidth,height=0.08\paperheight]{bgWave.pdf}}
 
\end{frame}

\begin{frame}{Shallow vs Deep Neural Networks}
 \begin{figure}[h]
  %\begin{wrapfigure}{r}{0.33\textwidth} %this figure will be at the right
  \centering
  %    \includegraphics[width=0.25\textwidth]{mesh}
  \includegraphics[scale=0.4]{figures/svsd.jpeg}
  %\end{wrapfigure}
 \end{figure}
 \begin{columns}
  \begin{column}{0.5\textwidth}
   \begin{itemize}
    \item one hidden layer
    \item easier training
   \end{itemize}
  \end{column}
  \begin{column}{0.5\textwidth}  %%<--- here
   \begin{itemize}
    \item several hidden layers
    \item more complex fuctions
   \end{itemize}
  \end{column}
 \end{columns}
 
\end{frame}

\begin{frame}{AUC comparison}
 \small From~\cite{paper}
 \begin{figure}
  \centering
  \includegraphics[width=.4\textwidth]{figures/SNp.png} \includegraphics[width=.4\textwidth]{figures/DNp.png}\\
  \vspace{0.5cm}
  \footnotesize \begin{tabular}{lccc}
   \toprule
   Discovery significance & Low level   & High level  & Complete    \\
   \midrule
   $\text{SNN}$           & $2.5\sigma$ & $3.1\sigma$ & $3.7\sigma$ \\
   $\text{DNN}$           & $4.9\sigma$ & $3.6\sigma$ & $5.0\sigma$ \\
   \bottomrule
  \end{tabular}
 \end{figure}
\end{frame}

\begin{frame}{Cluster and HPC}
 \begin{columns}
  \begin{column}{0.5\textwidth}
   \begin{figure}[h]
    \includegraphics[scale=0.035]{figures/cluster.JPG}
    \includegraphics[scale=0.08]{figures/nvidia_logo.png}
   \end{figure}
  \end{column}
  \begin{column}{0.5\textwidth}  %%<--- here
   \begin{textblock*}{10cm}(6.5cm,2.2cm)
    \footnotesize
    Cluster:
    \begin{itemize}
     \item several machine nodes
     \item introduce network issues
     \item different programming logic
    \end{itemize}
    \vspace{2.4cm}
    High performance computing with GPUs:
    \begin{itemize}
     \item much more (slower) cores than CPUs
     \item well suited for parallel tasks
    \end{itemize}
   \end{textblock*}
  \end{column}
 \end{columns}
\end{frame}


\begin{frame}{Software utilized}
 \begin{figure}[h]
  %\begin{wrapfigure}{r}{0.33\textwidth} %this figure will be at the right
  \centering
  %    \includegraphics[width=0.25\textwidth]{mesh}
  \includegraphics[scale=1]{figures/logos}
  %\end{wrapfigure}
 \end{figure}
\end{frame}

\begin{frame}{Performance comparison}
 \begin{textblock*}{15cm}(-1cm,1cm)
  \begin{figure}[h]
   \includegraphics[scale=0.2103]{figures/absolute.png}
   \includegraphics[scale=0.2103]{figures/relative.png}
   \caption{Number of nodes versus execution time for $1$ epoch}
   \label{1}
  \end{figure}
 \end{textblock*}
 
 \begin{textblock*}{10cm}(1.5cm,6.2cm)
  \begin{itemize}
   \item as the number of nodes increases, the choice of the algorithm becomes less important
   \item the bottleneck could be on the network side
   \item GPUs are less performing than expected
  \end{itemize}
 \end{textblock*}
\end{frame}

\begin{frame}{Outlook}
 These paradigms can give a significant boost to deep neural networks
 training, further exploration could be:
 \begin{itemize}
  \item set up a GPUs cluster given their different network technology
  \item work on network management to improve it
 \end{itemize}
\end{frame}

\section{First section}

\begin{frame}{$ $ }
 \centering
 \Huge Grazie per l'attenzione!
\end{frame}


\end{document}
